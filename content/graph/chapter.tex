\chapter{Graph}

% \section{Fundamentals}
	\kactlimport{BellmanFord.h}
	\kactlimport{FloydWarshall.h}
	% \kactlimport{TopoSort.h}

\section{Network flow}
	\kactlimport{Dinic.h}
	% \kactlimport{PushRelabel.h}
	% \kactlimport{EdmondsKarp.h}
	\kactlimport{MCMF-SPFA.h}
	\kactlimport{MinCostMaxFlow.h}
	% \kactlimport{MinCut.h}
	\kactlimport{GlobalMinCut.h}
	% \kactlimport{GomoryHu.h}

\section{Matching}
	% \kactlimport{hopcroftKarp.h}
	% \kactlimport{DFSMatching.h}
	% \kactlimport{MinimumVertexCover.h}
	\kactlimport{WeightedMatching.h}
	% \kactlimport{GeneralMatching.h}

\section{DFS algorithms}
	\kactlimport{SCC.h}
	\kactlimport{BiconnectedComponents.h}
	\kactlimport{2sat.h}
	% \kactlimport{EulerWalk.h}

\section{Coloring}
	\kactlimport{EdgeColoring.h}\
	% \kactlimport{BipartiteGraphColoring.h}\
	\kactlimport{MinimumBipartiteEdgeColoring.h}\
	

% \section{Heuristics}
% 	\kactlimport{MaximalCliques.h}
% 	\kactlimport{MaximumClique.h}
% 	\kactlimport{MaximumIndependentSet.h}

\section{Trees}
	\kactlimport{BinaryLifting.h}
	\kactlimport{LCA.h}
	\kactlimport{CompressTree.h}
	\kactlimport{HLD.h}
	% \kactlimport{LinkCutTree.h}
	% \kactlimport{DirectedMST.h}
	\kactlimport{CentriodDecomposition.h}
\section{Math}
\textbf{Number of Spanning Trees}
% I.e. matrix-tree theorem.
% Source: https://en.wikipedia.org/wiki/Kirchhoff%27s_theorem
% Test: stress-tests/graph/matrix-tree.cpp
Create an $N\times N$ matrix \texttt{mat}, and for each edge $a \rightarrow b \in G$, do
\texttt{mat[a][b]--, mat[b][b]++} (and \texttt{mat[b][a]--, mat[a][a]++} if $G$ is undirected).
Remove the $i$th row and column and take the determinant; this yields the number of directed spanning trees rooted at $i$
(if $G$ is undirected, remove any row/column).

\textbf{Erdős–Gallai theorem}
% Source: https://en.wikipedia.org/wiki/Erd%C5%91s%E2%80%93Gallai_theorem
% Test: stress-tests/graph/erdos-gallai.cpp
A simple graph with node degrees $d_1 \ge \dots \ge d_n$ exists iff $d_1 + \dots + d_n$ is even and for every $k = 1\dots n$,
\[ \sum _{i=1}^{k}d_{i}\leq k(k-1)+\sum _{i=k+1}^{n}\min(d_{i},k). \]

\textbf{Mirsky's Theorem} Max length chain is equal to min partitioning into antichains. Max chain is height of poset.

\textbf{Dilworth's Theorem} Min partition into chains is equal to max length antichain. From poset create bipartite graph. Any edge from $v_{i}$ - $v_{j}$ implies  $LV_{i}$ - $RV_{j}$. Let A be the set of vertices such that neither $LV_{i}$ nor $RV_{i}$ are in vertex cover. A is an antichain of size n-max matching. To get min partition into chains, take a vertex from left side, keep taking vertices till a matching exist. Consider this as a chain. Its size is n - max matching.

% \textbf{Matrix-tree Theorem} Let matrix $T = [t_{ij}]$, where $t_{ij}$ is negative of the number of multiedges between $i$ and $j$, for $i \ne j$, and $t_{ii} = \mbox{deg}_i$. Number of spanning trees of a graph is equal to the determinant of a matrix obtained by deleting any $k$-th row and $k$-th column from $T$. If $G$ is a multigraph and $e$ is an edge of $G$, then the number $\tau(G)$ of spanning trees of $G$ satisfies recurrence $\tau(G) = \tau(G-e) + \tau(G/e)$, when $G-e$ is the multigraph obtained by deleting $e$, and $G/e$ is the contraction of $G$ by $e$ (multiple edges arising from the contraction are preserved.)

% \textbf{Cycle Spaces} The (binary) cycle space of an undirected graph is the set of its Eulerian subgraphs. This set of subgraphs can be described algebraically as a vector space over the two-element finite field. One way of constructing a cycle basis is to form a spanning forest of the graph, and then for each edge $e$ that does not belong to the forest, form a cycle C  $C_{e}$ consisting of $e$ together with the path in the forest connecting the endpoints of $e$. The set of cycles $C_{e}$ formed in this way are linearly independent (each one contains an edge $e$ that does not belong to any of the other cycles) and has the correct size $m - n + c$ to be a basis, so it necessarily is a basis. This is fundamental cycle basis.

% \textbf{Cut Spaces} The family of all cut sets of an undirected graph is known as the cut space of the graph. It forms a vector space over the two-element finite field of arithmetic modulo two, with the symmetric difference of two cut sets as the vector addition operation, and is the orthogonal complement of the cycle space. To compute the basis vector for the cut space, consider any spanning tree of the graph. For every edge $e$ in the spanning tree, remove the edge and consider the cut formed. Thus dimension of the basis vector for cut space is n-1.

% \textbf{Number of perfect matchings of a bipartite graph} is equal to the permanent of the adjacency matrix obtained. To check the parity of the number of perfect matchings, we can evaluate the permanent of the matrix in $Z_{2}$ which can be done easily as Permanent(A) = Determinent(A).

% \textbf{Tutte Matrix}. For a simple undirected graph $G$, Let $M$ be a matrix with entries $A_{i, j} = 0$ if $(i, j) \notin E$ and $A_{i, j} = -A_{j, i} = X$ if $(i, j) \in E$. $X$ could be any random value. If the determinants are non-zero, then a perfect matching exists, while other direction might not hold for very small probability.

% \textbf{Kirchhoff's Theorem}. For a multigraph $G$ with no loops, define Laplacian matrix as $L = D - A$. $D$ is a diagonal matrix with $D_{i, i} = deg(i)$, and $A$ is an adjacency matrix. If you remove any row and column of $L$, the determinant gives a number of spanning trees.

% \textbf{Brook's Theorem} If a graph is not a complete graph or an odd cycle then it can be coloured with max degree \# of colours.

% \textbf{Turan's Theorem} A graph without $K_{r+1}$ and N verticies can have atmax $\lfloor \frac{N^{2}}{2} * (1 - \frac{1}{r}) \rfloor$ edges. 
